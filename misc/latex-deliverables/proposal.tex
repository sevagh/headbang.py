\documentclass[letter,12pt]{report}
%\setlength{\parindent}{0pt}
\usepackage[left=2cm, right=2cm, top=2cm, bottom=2cm]{geometry}
\usepackage[shortlabels]{enumitem}
\usepackage{graphicx}
\usepackage{amsmath}
\usepackage{amssymb}
\usepackage{verbatim}
\usepackage{listings}
\usepackage{minted}
\usepackage{subfig}
\usepackage{titling}
\usepackage{caption}
\setlength{\droptitle}{1cm}
\usepackage{hyperref}
\hypersetup{
    colorlinks,
    citecolor=black,
    filecolor=black,
    linkcolor=black,
    urlcolor=black
}
\usepackage{setspace}
\renewcommand{\topfraction}{0.85}
\renewcommand{\textfraction}{0.1}
\renewcommand{\floatpagefraction}{0.75}
\usepackage[backend=biber,authordate]{biblatex-chicago}
\addbibresource{citations.bib}
\usepackage{titlesec}
 
\titleformat{\chapter}[display]
  {\normalfont\bfseries}{}{0pt}{\Huge}

\begin{document}

\noindent\Large{\textbf{headbang.py}}\\
\large{Final project proposal. MUMT 621, March 30, 2021}\\
\large{Sevag Hanssian, 260398537}

\noindent\hrulefill

\vspace{2em}

The act of headbanging is when metal musicians or fans violently move their head up and down to the beat of a metal song. My final project will be a Python software project, headbang.py, which jointly studies audio beat tracking and headbanging motion analysis in metal music.

Beat tracking is a rich field of music information retrieval (MIR). The audio beat tracking task has been a part of MIREX since 2006,\footnote{\url{https://www.music-ir.org/mirex/wiki/2006:Audio_Beat_Tracking}} and receives submissions every year. Most recently, \textcite{bock1, bock2} have achieved state of the art results in their open-source madmom Python library (\cite{madmom}). The MIREX evaluation includes musically diverse and challenging datasets (\cite{beatmeta}). In my own experiments I noticed that the beat locations output by the best algorithms did not align with my personal urge to headbang in rhythmically complex progressive metal (\cite{meshuggah, periphery}). Beats are always output, even in silent, non-percussive, or soft parts of the song. Sometimes there are dense clusters of beat locations that are very close to each other and are impossible to headbang to (without hurting my neck). Finally, sometimes beat locations feel perceptually incorrect, such that they're not coincident with what I would call a beat.

For the first goal of my final project, I propose to explore various audio beat tracking algorithms and pre-processing techniques to predict beat locations in progressive metal songs that align closer with how a person would headbang -- namely, the beats should be more sparse and spread out to account for comfortable headbanging, and should align with high-energy percussive events.

\textcite{groove} state that ``whenever listeners have the impulsion to bob their heads in synchrony with music, the groove phenomenon is at work.'' Other recent papers have used 2D human pose and motion estimation to associate dance movements with musical beats (\cite{pose1, pose2}). For the second goal of headbang.py, I propose to analyze headbanging motion in metal videos with the OpenPose 2D human pose estimation library (\cite{openpose}). The results of motion analysis can be displayed alongside the results of audio beat tracking, to compare and contrast both phenomena.

\vfill
\clearpage

%\nocite{*}
\printbibheading[title={\vspace{-3.5em}References},heading=bibnumbered]
\vspace{-1.5em}
\printbibliography[heading=none]

\end{document}
